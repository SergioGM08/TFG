\chapter{Introducción} \label{Capitulo 1}
Este primer capítulo se centra en presentar el marco contextual en el que se desarrolla este trabajo. Por un lado, 

\section{Motivación y objetivos}


\section{Contexto y antecedentes del trabajo}


\subsection{Problemas de optimización}

\subsection{Algoritmos heurísticos}

\subsection{Métodos de optimización de Monte-Carlo}



\section{Estructura de la memoria}
La presente memoria se estructura del modo siguiente:

\begin{itemize}

\item En el presente \textbf{Capítulo \ref{Capitulo 1}} hemos presentado el tema de este trabajo, a través de 

\item El \textbf{Capítulo \ref{Capitulo 2}} se centra en 

\item En el \textbf{Capítulo \ref{Capitulo 3}} hablamos sobre  

\item El \textbf{Capítulo \ref{Capitulo 4}} se centra en 

\item Por último, en el \textbf{Capítulo \ref{Capitulo 5}} se presentan las conclusiones finales, el trabajo futuro y algunos trabajos publicados relacionados con la investigación realizada.

\end{itemize}

A lo largo de esta memoria se presentan diferentes cuadros de código que sirven para apoyar las experimentaciones realizadas y las explicaciones textuales. Sin embargo, la totalidad del código implementado puede encontrarse en el siguiente repositorio de Github \url{https://github.com/mavice07/TFG-Mates.git}.

Ejemplo de cita para lla bibliografía \cite{Sanchez2022}
